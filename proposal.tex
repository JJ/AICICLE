\documentclass[a4paper]{article}

%% Language and font encodings
\usepackage[english]{babel}

\usepackage{booktabs}

%% Sets page size and margins
%\usepackage[a4paper,top=2cm,bottom=2cm,left=3cm,right=3cm,marginparwidth=1.75cm]{geometry}

%% Useful packages
\usepackage{amsmath}
\usepackage{graphicx}
%\usepackage{apacite}
\usepackage[colorinlistoftodos]{todonotes}
\usepackage[colorlinks=true, allcolors=blue]{hyperref}
\usepackage[a4paper,top=3cm,bottom=2cm,left=2cm,right=2cm,marginparwidth=1.75cm]{geometry}

\title{PAIPLE: (Powered by AI) Personal Learning Environment}
\author{Juan J. Merelo Guervós}
\date{}

\begin{document}
\maketitle

\section{PI's CV}

Juan J. Merelo Guervós was born on March 10, 1965. He obtained a degree in Physics (major in Theoretical Physics) by the university of Granada in 1988, and a PhD in Physics by the same university in 1994. He is currently registered as a student in a BA in Art history in the same university, starting in 2024-2025.

He is full professor at the Department of Computer Engineering, Automatics and Robotics at the University of Granada, position obtained in 2009. He started teaching at the same university, at the department then called Electronics and Computer Technology, in 1988, going through a tenured position as adjunct professor in 1991, and assistant professor in 1998. His teaching experience in higher education is, thus, 36 years at the moment.

He is also an experience free software developer, as well as performed consulting jobs for industry during all his career, latest one as Senior Software Engineer for the German startup polypoly from May 2021 to October 2022.

An industry-oriented version of this CV can be obtained from \url{https://jj.github.io/cv}. A more academically oriented (and automatically updated) version at \url{https://github.com/JJ/cv/actions/workflows/generate-cv.yml}.

\subsection{Research}

An active researcher since 1988 as shown in his Google Scholar profile \url{https://scholar.google.com/citations?user=gFxqc64AAAAJ}, this profile includes several hundreds of publications, an $h$ index of 43 and 222 publications with more than 10 citations (58 in the last 10 years). The total number of citations is close to 10000 (9689 at the time of writing). This shows a continuous commitment for excellence in research and development, and research interests in areas that are continuously relevant.

These areas include currently application of statistical techniques to the study of history, study of complex systems (an area where he has been studing continuously for the past 30 years), evolutionary algorithms (also dating back to his PhD, more than 30 years back), including lately application of green computing principles to evolutionary algorithms with the objective of reducing the amount of energy necessary for workloads that include this kind of algorithms. His research in the area of artificial intelligence goes back to 1993, when he started to use neural networks for predicting the secondary structure of proteins. The resulting paper, ``Evaluation of secondary structure of proteins from UV circular dichroism spectra using an unsupervised learning neural network'', was a keystone in bioinformatics research with 1301 citations at the time of writing, 13 of them last year.

The highlight of his research in complex networks is the paper ``A network analysis of the 2010 FIFA World Cup champion team play'', which in 2013 applied complex network analysis to the game of soccer. It has received 168 citations, 13 of them in the last year.

He has been an active principal investigator of research projects since 1997,
including 3-year Spanish national research projects, as well as European funded
research projects and capacity-building projects such as EduBots, researching
the use of chatbots for improving the learning experience in higher
education. As a result of this project, the paper ``Chatbots and messaging
platforms in the classroom: An analysis from the teacher’s perspective'' was
published and immediately received attention by the education research
community, obtaining 14 citations in its first year in public. His research in
education dates back to 2004, with papers published on the use of wikis (2005) and
blogs in class (2004), and lately organizing a whole cloud computing class around
GitHub and agile workflows using formative evaluation (since 2014; previously we
used Google Code).

During his career, he has given tutorials in many research conferences, mainly
related to evolutionary algorithms: Parallel Problem Solving from Nature (PPSN),
World Conference on Computational Intelligence (WCCI), General Evolutionary
Computation Conference (GECCO). He has also been keynote speaker in conferences
such as Evolutionary Computation Theory and Applications (ECTA), Intelligent
Distributed Computing (IDC), GAMEON, Workshop on Engineering Applications (WEA),
1st International Conference on Artificial Intelligence: Theories and
Applications (ICAITA), Annual Workshop on Software Engineering.

{\em In general}, his publication record reflect a steady career in research
which is still ongoing, as well as a proved experience in the coordination and
direction of research grants and in innovation in education.

\subsection{Software development and engineering}

His experience in programming goes back to 1983, starting with BASIC, to proceed
with Pascal. However, professional exposure to languages started with C and C++
in the early 90s, as well as Perl and Python; JavaScript in the late 90s. Later
languages include professional proficiency in Raku and R, as well as some
experience with Kotlin, Rust, Zig, Go and other languages like Ruby and Lua.

The longest and still running experience is with Perl. He has been using it
since the early 90s, and is a contributor of 19 modules to MetaCPAN
\url{https://metacpan.org/author/JMERELO}, most of them actively maintained. He
has also participated in many different Perl conferences, since Yet Another Perl
Conference in Münich in 2002, including Perl workshops in London, the
Netherlands and Switzerland.

He has been also part of the core contributors of the Raku (née Perl 6)
language, receiving a grant from the Perl foundation to orgenize the
documentation, and participating ever since in the language, focused mainly on
the documentation, but also being an elected member to the Steering Committee
since its inception and for several years. Also keeps maintaining 29
distributions in the Raku module ecosystem
\url{https://raku.land/zef:jjmerelo}. He was also in the effort to include Perl
and Raku in Google Summer of Code, and mentor of two of the chosen projects,
participating as such in the Google Summer of Code Mentor Summit of the same
year, 2019.

Most popular open source product, however, is a GitHub action that checks pull
requests for presence or absence of certain patterns. This action
\url{https://github.com/marketplace/actions/check-pull-request-body-diff-and-files}
has 28 stars in GitHub, and circa 200 repositories are using it, some of them as
popular as W3C's {\tt aria} or the reading server {\tt kavita}.

He is an early contributor to GitHub (as shown by his nick {\tt JJ}, as well as
prolific, being consistently among the top GitHub contributors by number of
commits in Spain
\url{https://github.com/gayanvoice/top-github-users/}; this is mainly due to a
policy of open class materials, open science, and free software, with everything
from class assignments to scientific papers carried out openly in GitHub.

Professional experience is mainly linked to consulting jobs invoiced through the
university of Granada, of which the latest is the part-time senior software
engineer job in polypoly, that lasted from May 2021 to October 2022.

{\em Summarizing}, he has got sufficient experience to specify and lead a development
team or participate in it, as well as carry out programming tasks proficiently
in the languages that have been mentioned, or start to use rapidly new
languages, methodologies or techniques.

\subsection{Prizes and awards}

In two separate years, he has received the Google Computer Science for High
School award, in 2014 and 2017, in the first case it was for the organization of
a computer science campus for high school girls, which has been organized at the
university of Granada ever since, creating a sustainable and ongoing effort; in
the second case, it was called Computing4Life and it was addressed mainly at
educators who would be able to transmit interest in computer science to
students.

{\em Summarizing}, he has got a track record of successfully being awarded and
administering Google grants.

\newpage

\section{Proposal}

\bibliographystyle{plain}
\bibliography{paiple}

\end{document}