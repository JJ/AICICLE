\documentclass[a4paper,12pt]{article}

%% Language and font encodings
\usepackage[english]{babel}


%% Sets page size and margins
%\usepackage[a4paper,top=2cm,bottom=2cm,left=3cm,right=3cm,marginparwidth=1.75cm]{geometry}

%% Useful packages
\usepackage{graphicx}
%\usepackage{apacite}
\usepackage[colorlinks=true, allcolors=blue]{hyperref}
\usepackage[a4paper,top=3cm,bottom=2cm,left=1in,right=1in]{geometry}

\title{PAIPLE: (Powered by AI) Personal Learning Environment}
\author{Juan J. Merelo Guervós\\
\footnotesize Department of Computer Engineering, Automatics
and Robotics, ETSIIT, University of Granada\\
\footnotesize C/ Daniel Saucedo
  Aranda, s/n, 18071 Granada, Spain}
\date{}

\begin{document}


\maketitle

\abstract{A successful tool needs to be centered on students in order to improve
  learning outcomes at scale. In this proposal, Power of AI to the PeoPLE, we
  will develop an AI-enhanced personal learning environment that helps the
  student improve retention of any subject, giving them the tools to succeed
  independently of the evaluation environment. 
}

\section{Research goals and problem statement}

Innovations in educations are traditionally teacher and institution oriented and
driven. Teachers prefer this kind of approach since an institutional approach is
essential for sustainability \cite{DBLP:journals/eait/MereloCMBAGT24}, while
institutions prefer this at the allocation of resources, computational and
otherwise. However, from the point of view of the student, there are two
problems: first is it moves at the speed of institutions, and second the
top-down approach may fail to address student concerns. From the purely
pragmatic point of view of a researcher, deployment of new strategies that need
the concourse of teachers, institutions and students with no stakes are probably
doomed from the start.

Focusing on student needs, on the other hand, present a different
perspective. Often, students face out-of-date methodologies, high-stakes written
exams, as well as bibliographies and concepts that might not be up-to-date with
the latest information, and this happens across the board, from scientific to
humanistic through technical disciplines. At any rate, it is a learning
environment that can have a high level of exigence, without providing, at the
institutional or technical level, the tools to deal with it.

However, student-centered education has been proved to improve learning
outcomes\cite{kaput2018evidence} with respect to traditional pedagogies. In an
environment with abundant information sources, giving autonomy to the student
improves learning outcomes across the board, independently of the form of
evaluation and the type of subject. The main research goal is to improve
autonomy and retention of information of students in tertiary education via
AI-enhanced personal learning environments. The secondary goal is to test the
efficiency of large and small language models in context-restricted learning
environments, as well as personal-device based open-source models vs. cloud
based, public (i.e. without the need of institutional support) models such as
Gemini.

\section{Description of the proposed work, expected outcomes, and results}

What we expect to deliver by the end of this project is an open source
application that works as a class-specific personal learning environment, and
that the student is able to easily install and carry wherever they want.

It is a fact that student nowadays use extensively generative AIs to carry out
class assignments as well as essays; most students are familiarized with the
free version of any of them. This use is more task-oriented than
learning-oriented: the learning objective is thus usually removed from the
equation. On the other hand, it certainly does not help when you have to
assimilate information related to one subject and deliver it under demand in a
written or oral exam.

Despite the fact that Personal Learning Environments have been described for
several decades, and deployed extensively, generally in a top-down fashion as
Learning Management Systems, its use among the student population is marginal
\cite{serhan2022systematic}, and generally limited to storing class material
ellaborated by professors, without any further learning-oriented
ellaboration.

On the other hand, very recent research \cite{ruiz2024learning} has proved that
several techniques such as retrieval practice, ellaboration and spacing help
learning outcomes. What we propose is to use generative AI, fed by contextual
material and incorporated in the PLE, so that the student is able to receive
inmediate feedback on the information that has been retrieved, ellaborated
information, as well as prompts to organize study in periodic study
sessions. The PLE can give feedback on specific errors in these techniques, as
well as a study guide for the parts of the material or learning objectives that
have not been covered in the retrieval or ellaboration exercises. This is
similar to what is called ungrading \cite{10.1145/3587102.3588816}, but without
needing to include the professor in the loop or engage their collaboration.
Additionally, this can be helped by exams elaborated and graded by the AI in the
same shape (multiple choice, short essay) that the student will be actually
sitting for later on, a practice that has also been proved helpful for students
\cite{YANG20197324}.

We will perform the study in different phases, using always groups of at least
25 students in every group under study. We will use student boards in our
university to announce the study, as well as in other universities of colleagues
that are willing to participate. We will try to cover as many disciplines as
possible, with as many different kind of disciplines as possible. Student
collaboration will be encouraged via Amazon (or other) gift vouchers, to be
collected when the survey at the end of the semester is filled correctly and
telemetry indicates that the use of the facility has been correct. We will organize research in two phases\begin{itemize}
  \item In a first phase, that will run in parallel to software development,
    students will be using no PLE at all, or following a manual that we will
    publish on how to use Gemini or other LLM to improve learning in any
    subject. This will be used as baseline. When possible, whole classes will be
    recruited, and we will encourage professors with a 100€ gift voucher. During
    this phase, a Google Workspace based solution will be developed; this is a
    familiar tool already deployed in many campuses, and with which many
    students already work. This solution will use GW to store class material
    either teacher-provided or found by the student, and will feed it to Gemini
    to create class-specific models. These will be included in a Google
    Workspace add-on, which will include calendaring as well as ellaboration
    grading and retrieval practice with feedback.
  \item The second phase will effectively use this Google Workspace add-on,
    available through the Add-on marketplace in beta version, essentially
    looking at students giving us feedback at the end of the exam period on the
    results achieved. Making it available through the marketplace will enhance
    adoption, and the fact that it will also be released as free software will
    help create a community of developers around it that will be able to help or
    even drive innovation in this area, adding new capabilities. We will carry
    out this phase through 2 semesters, one in alpha and the other in beta.
  \end{itemize}

Software architecture will be organized in such a way that document as well as
LLM providers are isolated, in such a way that it can be ported to langchains
easily, if possible (and if resources are available) within the project
itself. At any rate, the final deliverable will be a minimally viable product
released as a Google Workspace add-on as well as in a GitHub repository with an
Affero GPL license.

This line of research continues what was initiated with the EduBots project,
whose node in Granada was coordinated by the proposer
\cite{DBLP:journals/eait/MereloCMBAGT24,DBLP:conf/hci/MoraGBCG21}. Lessons
learned in software development as well as in the deployment of innovations in
the classroom will be incorporated into this research. It also complements work
in student-centered teaching recently published \cite{10481/89811}. It continues
research on use of language models in education initiated recently by other
researchers, such as \cite{li2024adaptinglargelanguagemodels} but, over all, it
tries to put in practice insights published  very recently by the already mentioned
\cite{ruiz2024learning} in a way that can help potentially any student in the world.


\section{Explanation of your qualifications to conduct this research}

As a person that is both a professor (in a Computer Science degree) and a
student (in a History of Art degree) he is in a unique position to understand
the problems of students from boths points of view, as learning outcomes with
which students arrive to his class, and as a student that needs to pass written
exams that count 70\% of the grade. This puts him in a position to recruit users
of the PAIPLE from both ends, and in very dissimilar disciplines, although of
course the students recruited for the study will not be limited to this
university or even this country.

On the other hand, he has a proven track record either in educational research
an innovation
\cite{DBLP:conf/ijcci/ValdezGHS17,DBLP:journals/eait/MereloCMBAGT24} or in
research in computational intelligence techniques such as neural networks,
including deep neural nets, and evolutionary algorithms, as well as development
of free software, thus being able to span the whole development process from
problem identification through solution deployment through experimental design
and analysis. Additionally, as shown in his recent list of publications, he
maintains collaborations with professors in different universities in 4
continents, whose collaboration can be requested to recruit new students.

\section{Data policy}

We follow a rather strict open science policy, including open notebook science,
publishing all research data in our Github repository:
\url{https://github.com/JJ}. This includes opening a repository per research
  deliverable, that gathers code, data, as well as papers that describe them.

Since we will be dealing with individual student outcomes, we will use an
unidirectional hash to anonymize their results and any other data gathered. All
personal data will be treated according to the European GDPR directive.

\newpage

\section{PI's CV}

Juan J. Merelo Guervós was born on March 10, 1965. He obtained a degree in Physics (major in Theoretical Physics) by the university of Granada in 1988, and a PhD in Physics by the same university in 1994. He is currently registered as a student in a BA in Art History in the same university, starting in 2024-2025.

He is full professor at the Department of Computer Engineering, Automatics and
Robotics at the University of Granada, position obtained in 2009. He started
teaching at the same university, at the department then called Electronics and
Computer Technology, in 1988, going through a tenured position as adjunct
professor in 1991, and assistant professor in 1998. His teaching experience in
higher education is, thus, 36 years at the moment. He is also an experienced
free and commercial software developer.

% An industry-oriented version of this CV can be obtained from \url{https://jj.github.io/cv}. A more academically oriented (and automatically updated) version at \url{https://github.com/JJ/cv/actions/workflows/generate-cv.yml}.

\subsection{Research}

An active researcher since 1988 as shown in his Google Scholar profile \url{https://scholar.google.com/citations?user=gFxqc64AAAAJ}, this profile includes several hundreds of publications, an $h$ index of 43 and 222 publications with more than 10 citations (58 in the last 10 years). The total number of citations is close to 10000 (9689 at the time of writing). This shows a continuous commitment for excellence in research and development, and research interests in areas that are currently relevant.

These areas include currently application of statistical techniques to the study of history, study of complex systems (an area where he has been studing continuously for the past 30 years), evolutionary algorithms (also dating back to his PhD, more than 30 years back), including lately application of green computing principles to evolutionary algorithms with the objective of reducing the amount of energy necessary for workloads that include this kind of algorithms. His research in the area of artificial intelligence goes back to 1993, when he started to use neural networks for predicting the secondary structure of proteins. The resulting paper, ``Evaluation of secondary structure of proteins from UV circular dichroism spectra using an unsupervised learning neural network'', was a keystone in bioinformatics research with 1301 citations at the time of writing, 13 of them last year.

The highlight of his research in complex networks is the paper ``A network analysis of the 2010 FIFA World Cup champion team play'', which in 2013 applied complex network analysis to the game of soccer. It has received 168 citations, 13 of them in the last year.

He has been an active principal investigator of research projects since 1997,
including 3-year Spanish national research projects, as well as European funded
research projects and capacity-building projects such as EduBots, researching
the use of chatbots for improving the learning experience in higher
education. As a result of this project, the paper ``Chatbots and messaging
platforms in the classroom: An analysis from the teacher’s perspective'' was
published and immediately received attention by the education research
community, obtaining 14 citations in its first year in public. His research in
education dates back to 2004, with papers published on the use of wikis (2005) and
blogs in class (2004), and lately organizing a whole cloud computing class around
GitHub and agile workflows using formative evaluation (since 2014; previously we
used Google Code).

% During his career, he has given tutorials in many research conferences, mainly
% related to evolutionary algorithms: Parallel Problem Solving from Nature (PPSN),
% World Conference on Computational Intelligence (WCCI), General Evolutionary
% Computation Conference (GECCO). He has also been keynote speaker in conferences
% such as Evolutionary Computation Theory and Applications (ECTA), Intelligent
% Distributed Computing (IDC), GAMEON, Workshop on Engineering Applications (WEA),
% 1st International Conference on Artificial Intelligence: Theories and
% Applications (ICAITA), Annual Workshop on Software Engineering.

{\em In general}, his publication record reflect a steady career in research
which is still ongoing, as well as a proved experience in the coordination and
direction of research grants and in innovation in education.

\subsection{Software development and engineering}

His experience in programming goes back to 1983, starting with BASIC, to proceed
with Pascal. However, professional exposure to languages started with C and C++
in the early 90s, as well as Perl and Python; JavaScript in the late 90s. Later
languages include professional proficiency in Raku and R, as well as some
experience with Kotlin, Rust, Zig, Go and other languages like Ruby and Lua.

He has been using Perl since the early 90s, and is a contributor of 19 modules
to MetaCPAN \url{https://metacpan.org/author/JMERELO}, most of them actively
maintained. He has also participated as speaker and organizer in many different
Perl conferences and tracks. He has been also part of the core contributors of
the Raku (née Perl 6) language, receiving a grant from the Perl Foundation, and
participating ever since in the language, focused mainly on the documentation,
and as elected member to the Steering Committee. Also keeps maintaining 29
distributions in the Raku module ecosystem
\url{https://raku.land/zef:jjmerelo}. He was also in the effort to include Perl
and Raku in Google Summer of Code, and mentor of two of the chosen projects,
participating as such in the Google Summer of Code Mentor Summit of the same
year, 2019.

% Most popular open source product, however, is a GitHub action that checks pull
% requests for presence or absence of certain patterns. This action
% \url{https://github.com/marketplace/actions/check-pull-request-body-diff-and-files}
% has 28 stars in GitHub, and circa 200 repositories are using it, some of them as
% popular as W3C's {\tt aria} or the reading server {\tt kavita}.

% He is an early contributor to GitHub (as shown by his nick {\tt JJ}, as well as
% prolific, being consistently among the top GitHub contributors by number of
% commits in Spain.
% \url{https://github.com/gayanvoice/top-github-users/}; this is mainly due to a
% policy of open class materials, open science, and free software, with everything
% from class assignments to scientific papers carried out openly in GitHub.

Professional experience is mainly linked to consulting jobs invoiced through the
university of Granada, of which the latest is the part-time senior software
engineer job in polypoly, that lasted from May 2021 to October 2022.

{\em Summarizing}, he has got sufficient experience to specify and lead a development
team or participate in it, as well as carry out programming tasks proficiently
in the languages that have been mentioned, or start to use rapidly new
languages, methodologies or techniques.

\subsection{Prizes and awards}

In two separate years, he has received the Google Computer Science for High
School award, in 2014 and 2017, in the first case it was for the organization of
a computer science campus for high school girls, which has been organized at the
university of Granada ever since, creating a sustainable and ongoing effort; in
the second case, it was called Computing4Life and it was addressed mainly at
educators who would be able to transmit interest in computer science to
students.

{\em Summarizing}, he has got a track record of successfully being awarded and
administering Google grants.

\subsection{Other merits}

He has published two books on Raku/Perl 6 under the Apress seal, ``Learning Perl
6'' and ``Raku cookbook''. He has also published several novels and essays, in
Spanish and English, having received a literary prize for his novel
``lujoyglamour.net''.

His teaching material and tutorials, mainly on complex networks and Perl, have
been used thoroughly by students. A book, self published in collaboration on
git, is a long-seller in Amazon throughout all its shops since it was published
several years ago and has an average 4 star valoration; this book is also
available from its repository.



\bibliographystyle{plain}
\bibliography{paiple}

\end{document}